%% LyX 2.3.4.3 created this file.  For more info, see http://www.lyx.org/.
%% Do not edit unless you really know what you are doing.
\documentclass[twocolumn,english]{extarticle}
\usepackage{helvet}
\usepackage{courier}
\renewcommand{\familydefault}{\sfdefault}
\usepackage[T1]{fontenc}
\usepackage[latin9]{inputenc}
\usepackage[letterpaper]{geometry}
\geometry{verbose,tmargin=1in,bmargin=1in,lmargin=0.65in,rmargin=0.65in}
\usepackage{babel}
\usepackage{amsmath}
\usepackage{amsthm}
\usepackage[unicode=true]
 {hyperref}

\makeatletter
%%%%%%%%%%%%%%%%%%%%%%%%%%%%%% Textclass specific LaTeX commands.
\numberwithin{equation}{section}
\numberwithin{figure}{section}

%%%%%%%%%%%%%%%%%%%%%%%%%%%%%% User specified LaTeX commands.
\usepackage[anythingbreaks]{breakurl}

\makeatother

\usepackage{listings}
\renewcommand{\lstlistingname}{Listing}

\begin{document}
\title{Sandboxing in Operating Systems}
\author{Austin Kim}
\maketitle
\begin{abstract}
Sandboxing is one of many critical components of operating systems
protection. It allows for the partitioning of system resources and
the practice of executing processes with the least amount of privilege
required to accomplish tasks. This strict partitioning ensures that
even if a process is malicious, its ability to do damage on the system
at large is limited. Sandboxing is a part of the larger defense-in-depth
philosophy of operating tsystem protection concept. 
\end{abstract}

\section{Sandboxing basics}

\emph{Sandboxing} is the practice of placing immutable restrictions
on a process's access to resources \cite{key-1}, such as system calls,
privileges, storage space, and many others. This creates a set of
rigid access rules that cannot be disobeyed and creates a secure environment
upon which processes may be executed.

To accomplish sandboxing, the access matrix\footnote{An abstract model composed of domains and system objects and what
kind of access each domain has on each object.} is partitioned into ``horizontal'' slices (by domain) and then
conferred to the appropriate processes \cite{key-3}, often as early
as before the process's \lstinline[basicstyle={\ttfamily}]!main()!
function. This ensures that the restrictions placed upon it cannot
be changed at a later time.

It is also within the realm of possibility to sandbox processes and
limit their access to system calls to a small set. For example, a
process that only has read privileges but is not allowed to invoke
the \lstinline[basicstyle={\ttfamily}]!write()! call cannot modify
any of the disk contents allocated to it, effectively isolating it
from communicating with other processes.

\section{Why bother?}

Sandboxing is an important component in an operating system's protection
scheme, as it allows for \emph{per-process} control over what they
can and cannot access. 

Each time a new process is forked from its parent, it inherits the
same permissions that the parent held \cite{key-2}. The dangers of
a rogue process inheriting its parent's permissions is as blatant
as it is debilitating. Sandboxing allows the operating system to take
charge and section off process's access based on their needs such
that even if a malicious piece of code is executed, the damage done
to the system is either minimal or nonexistent. 

\section{Sandboxing examples}

At the operating systems level, sandboxing has been a feature since
many decades prior. In the beginning, there was the \emph{namespace}
feature in \emph{Plan 9 from Bell Labs}. In Unix, there was \lstinline[basicstyle={\ttfamily}]!chroot!
for isolating processes into a confined environment and still exists
to this day. In modern day Linux, \lstinline[basicstyle={\ttfamily}]!cgroups!
became the de-facto tool for limiting resource access for processes.
Many others, from full-scale virtual machines for sandboxing at the
OS-level, to containers like \emph{Docker} and \emph{systemd-nspawn},
sandboxing remains a staple of process compartmentalization and a
general protection feature.

There are numerous examples of sandboxing -- too many to list them
all. In this section, we will delve into some prominent sandboxing
examples in Linux: \lstinline[basicstyle={\ttfamily}]!chroot!, \lstinline[basicstyle={\ttfamily}]!cgroups!,
and \lstinline[basicstyle={\ttfamily}]!seccomp!.

\subsection{chroot}

\lstinline[basicstyle={\ttfamily}]!chroot! is a lightweight sandboxing
solution available to Linux, whose original Unix version was the predecessor
for the much more capable \lstinline[basicstyle={\ttfamily}]!jail!
in FreeBSD.

\lstinline[basicstyle={\ttfamily}]!chroot! uses Linux namespaces
to accomplish sandboxing. Linux namespaces, derived from \emph{Plan
9 From Bell Labs}'s namespaces \cite{key-2}, are a kernel feature
where every global resource within the file system follows a unified
namespace scheme, but those names point to completely different locations
per process \cite{key-8}. As such, \lstinline[basicstyle={\ttfamily}]!chroot!'s
sandboxing is largely limited to file access and is unable to compartmentalize
the process away from things like network sockets.

To see how \lstinline[basicstyle={\ttfamily}]!chroot! acts as a namespace-based
sandbox, it's worthwhile to take a look at the generalized steps for
setting up an environment for one \cite{key-9}: 
\begin{enumerate}
\item Create a top level directory to act as the root.
\item Copy over or symlink the essential directories necessary to run a
bare minimum environment (path to \lstinline[basicstyle={\ttfamily}]!sh!,
among others). This step is often automated by scripts.
\item Create symbolic links to any additional system resources that the
\lstinline[basicstyle={\ttfamily}]!chroot! environment should have
read/write access to, such as directories or files.
\item Run \lstinline[basicstyle={\ttfamily}]!chroot DIR CMD! to run a command
under this new environment.
\end{enumerate}
Steps 2 and 3 outline the gist of how \lstinline[basicstyle={\ttfamily}]!chroot!
works as a sandbox. It's trivial to select which shell the environment
should run with, which outside files it should be allowed to see,
and which files it can write to. Any write operations done inside
the environment is confined to the environment unless a link exists
to an outside directory that the process can write to. It is, in a
way, a very lightweight virtual machine with very little overhead.

This is not to say that \lstinline[basicstyle={\ttfamily}]!chroot!
is perfect at its job. If anything, it's one of the weakest sandboxing
tools available, albeit an easy one to use. There are many documented
and proven ways in which any user can escape the \lstinline[basicstyle={\ttfamily}]!chroot!
environment and gain direct access to all system resources \cite{key-10}. 

\subsection{cgroups}

Control groups, abbreviated \lstinline[basicstyle={\ttfamily}]!cgroups!,
are groups of processes that are bundled together into hierarchies
and their resource usage monitored. Following the ``everything is
a file'' philosophy of Unix and its derivatives, its interfaces are
exposed as a set of files in a pseudo-filesystem called \emph{cgroupfs}
\cite{key-11}. It is the underlying kernel feature upon which big-name
applications like \emph{Docker} are built. 

\lstinline[basicstyle={\ttfamily}]!cgroups! consists of two main
pieces:
\begin{itemize}
\item \emph{cgroup} bundles -- a collection of processes with limits imposed
upon them with regards to resource usage.
\item a \emph{subsystem} (also known as \emph{resource controllers} or \emph{controllers})
-- a kernel component that modifies the behavior of the processes
in a cgroup. 
\end{itemize}
The controller and the cgroup bundles are organized into a \emph{hierarchy}
that constantly changes the underlying cgroupfs subdirectories. It's
under these subdirectories that the access control is defined: whether
by adding processes to it by writing the pid to \lstinline[basicstyle={\ttfamily}]!cgroups.procs!,
or defining the process's behavior by assigning it to a specific subdirectory.

There are two different versions of \lstinline[basicstyle={\ttfamily}]!cgroups!:
\begin{itemize}
\item Cgroups v1 -- Released in 2008. Commonly mounted under \lstinline[basicstyle={\ttfamily}]!/sys/fs/cgroup!.
\item Cgroups v2 -- Attempts to unify the \lstinline[basicstyle={\ttfamily}]!cgroups!
standard and make the interface more consistent. Commonly mounted
under \lstinline[basicstyle={\ttfamily}]!/sys/fs/cgroup/unified!.
\end{itemize}
Both versions of \lstinline[basicstyle={\ttfamily}]!cgroups! may
be active at the same time, as v2 supports only a portion of controllers
found in v1. The only caveat is that the appropriate controllers be
used for different versions of \lstinline[basicstyle={\ttfamily}]!cgroups!.
For instance, using v1's controllers for v2's subdirectories would
not be allowed. 

\subsubsection{Creating and deleting cgroups}

Using \lstinline[basicstyle={\ttfamily}]!cgroups! is as easy as simply
writing to a file. For the sake of example, let's assume that \lstinline[basicstyle={\ttfamily}]!cgroups!
v1 is being used, and a process needs to be added under the \lstinline[basicstyle={\ttfamily}]!cpu!
controller subdirectory:
\begin{enumerate}
\item Create a directory for the new cgroup: \lstinline[basicstyle={\ttfamily},breaklines=true]!/sys/fs/cgroup/cpu/mygroup!.
This creates an empty cgroup named 'mygroup.'
\item Write the pid of the process to the \lstinline[basicstyle={\ttfamily}]!cgroups.procs!
file: \lstinline[basicstyle={\ttfamily},breaklines=true]!/sys/fs/cgroup/cpu/mygroup/cgroups.procs!.
\item The process of writing pid to \lstinline[basicstyle={\ttfamily}]!cgroups.procs!
automatically removes that process from all other cgroup subdirectories,
and assigns all threads belonging to that process to the cgroup immediately.
\end{enumerate}
Removing a cgroup is equally simple: Make sure that no active processes
belong under the cgroup, and simply remove the directory \cite{key-11}.

\subsubsection{Controller restrictions}

So, how exactly does \lstinline[basicstyle={\ttfamily}]!cgroups!
compartmentalize system resource usage? 

There are several predefined controller groups that are allowed certain
resources and priorities based on the directory structure. In the
above example, one of them (\lstinline[basicstyle={\ttfamily}]!cpu!)
was used to showcase how processes are assigned to individual controllers.
Any processes belonging to a controller group must abide by the rules
set for that controller group, and any descendant groups may not override
them \cite{key-11}. 

Listed below are some examples of v1 controllers:
\begin{enumerate}
\item \lstinline[basicstyle={\ttfamily}]!cpu! -- This controller group
is guaranteed to be assigned at least some CPU cycles even when the
system is busy.
\item \lstinline[basicstyle={\ttfamily}]!devices! -- This controller group
allows its processes to read and write to devices, such as disks.
\item \lstinline[basicstyle={\ttfamily}]!freezer! -- All processes and
their children in this controller group are allowed to be frozen at-will
depending on the need of the system.
\item \lstinline[basicstyle={\ttfamily}]!pids! -- Any processes belonging
to this controller group are limited in how many child processes they
may spawn.
\end{enumerate}

\subsection{seccomp}

\emph{Secure Computing state}, or \lstinline[basicstyle={\ttfamily}]!seccomp!,
is a kernel mode for processes that either limits said process to
a small subset of system calls, or limits them based on \emph{Berkeley
Packet Filters} (BPFs) passed to it \cite{key-12}. It is enabled
by the kernel config flag \lstinline[basicstyle={\ttfamily}]!CONFIG_SECCOMP=y!
at compile time.

\lstinline[basicstyle={\ttfamily}]!seccomp! has two main modes:
\begin{enumerate}
\item \lstinline[basicstyle={\ttfamily}]!SECCOMP_SET_MODE_STRICT! -- The
calling thread only has access to a subset of system calls consisting
of \lstinline[basicstyle={\ttfamily}]!read()!, \lstinline[basicstyle={\ttfamily}]!write()!,
\lstinline[basicstyle={\ttfamily}]!exit()!, and \lstinline[basicstyle={\ttfamily}]!sigreturn()!
only.
\item \lstinline[basicstyle={\ttfamily}]!SECCOMP_SET_MODE_FILTER! -- The
allowed list of system calls are passed by a pointer to a BFP filter. 
\end{enumerate}
Any process may ``elevate'' itself to one of these more secure modes
by invoking the \lstinline[basicstyle={\ttfamily}]!prctl()! function
during execution \cite{key-13}:
\begin{itemize}
\item \lstinline[basicstyle={\ttfamily},breaklines=true]!prctl(PR_SET_SECCOMP, SECCOMP_MODE_STRICT);!
for strict mode
\item \lstinline[basicstyle={\ttfamily},breaklines=true]!prctl(PR_SET_SECCOMP, SECCOMP_MODE_FILTER, filter);!
for filter mode 
\end{itemize}
Enforcement of \lstinline[basicstyle={\ttfamily}]!seccomp! is simple:
Any invocations of system calls outside the process's allowed list
results in the process being killed by the operating system. This
way, no matter how malicious the program is, its ability to do damage
is severely limited by the fact that it simply cannot issue the potentially
more damaging system calls. 

\section{Conclusion}

More than ever, the need to protect computer systems from malicious
code is ever present. The ability to fragment physical machines into
logical partitions that cannot easily broach the divide is indispensable
in a world driven by sensitive data stored in centralized servers.
To this end, sandboxing serves a critical purpose. 
\begin{thebibliography}{1}
\bibitem{key-1}Silberschatz, Abraham, et al. \emph{Operating System
Concepts}. 10th ed. Laurie Rosatone, 2018.

\bibitem{key-3}Tanerbaum, Andrew S., and Albert S. Woodhull. \emph{Operating
Systems Design and Implementation}. 3rd ed., Prentice Hall, 2006.

\bibitem{key-2}Ballesteros, Francisco J. \emph{Introduction to Operating
Systems Abstractions Using Plan 9 from Bell Labs}. Draft 9/28/2007.

\bibitem{key-8}``namespaces(7).'' The Linux man-pages project,
11 Apr. 2020, \href{http://man7.org/linux/man-pages/man7/namespaces.7.html}{http://man7.org/linux/man-pages/man7/namespaces.7.html}.

\bibitem{key-9}``chroot: Run a command with a different root directory.''
GNU Operating System. \href{https://www.gnu.org/software/coreutils/manual/html_node/chroot-invocation.html}{https://www.gnu.org/software/coreutils/manual/ html\_node/chroot-invocation.html}.
Accessed 18 Apr. 2020.

\bibitem{key-10}``How to break out of a chroot() jail.'' Simon's
computing stuff. \href{https://web.archive.org/web/20160127150916/http://www.bpfh.net/simes/computing/chroot-break.html}{https://web.archive.org/web/20160127150916/http:// www.bpfh.net/simes/computing/chroot-break.html}.

\bibitem{key-11}``cgroups(7).'' The Linux man-pages project, 11
Apr. 2020, \href{http://man7.org/linux/man-pages/man7/cgroups.7.html}{http://man7.org/linux/man-pages/man7/cgroups.7.html}.

\bibitem{key-12}``seccomp(2).'' The Linux man-pages project, 19
Nov. 2019, \href{http://man7.org/linux/man-pages/man2/seccomp.2.html}{http://man7.org/linux/man-pages/man2/seccomp.2.html}.

\bibitem{key-13}``Denying syscalls with seccomp.'' Eigenstate.
\href{https://eigenstate.org/notes/seccomp}{https://eigenstate.org/notes/seccomp}.
Accessed 19 Apr. 2020.
\end{thebibliography}

\end{document}
